\chapter{jcc}

jcc (Janus Circuit Compiler) is a compiler implemented for a modified version
of the Janus programing language\cite{YG:2007}.  It's code is available at
\url{https://github.com/aparent/jcc}.  It does not implement the full MDD
scheme but does provide the equivalent to eager cleanup at the statement level.

The compiler targets quantum circuits.

\section{Implementation}
\subsection{Gate Set}

The gate set used by the compiler includes the NOT, CNOT, and Toffoli gates.

When choosing the implementation of various operations consideration is given to the expansion into the Clifford+T gate set. More specifically to the minimization of T-gates. For example shared controls on Toffoli gates are desired as they result in T cancellation.

\subsection{Addition}
Addition is done using the CDKM\cite{CDKM:2004} adder as shown below:

\verbatiminput{jcc-examples/add.j}
\includegraphics[width=\textwidth]{images/add.pdf}

The final carry bit is not computed in this adder.
The result is an adder which computes $a+b \mod 2^n$

\subsection{Subtraction}
Subtraction can be done simply by reversing the addition circuit.

\subsection{Multiplication}
Multiplication is done with a simple shift and add circuit.
An adder adding the variables $a$ and $b$ consists of a CDKM\cite{CDKM:2004} adder controlled on each bit of a.
Each adder adds $b$ to some ancilla shifted left by the position of the control in a.
Each adder is smaller then the last as the multiplication is performed $\mod 2^n$.
Below is an example of the compiler output for a multiplication:

\verbatiminput{jcc-examples/mult.j}
\includegraphics[width=\textwidth]{images/mult.pdf}

\subsection{Conditional Statements}
Conditional if-else branches can be evaluated by swapping the bits to the correct circuit path controlled on the if conditional.
The other circuit path is evaluated on a set of ancilla bits is initialized to $H^{\otimes n}\ket{0}$.
Since this is an eigenvector of every permutation matrix it will be unchanged and can be cleaned up by reversing the initialization.

\verbatiminput{jcc-examples/ifExample.j}
\includegraphics[width=\textwidth]{images/ifExample.pdf}

Note that the assertion is a statement about the state of the program which is true if and only if the ``if'' branch is taken rather then the ``else'' branch.
This is useful as it describes a property of our data that can be used to clean up the bit used to store the result of the conditional.
i.e. If we know that we took that if branch rather then the else branch we can XOR the conditional bit and be assured it is being reset to zero.


\subsection{Loops}
In the circuit model we need to implement all loop operations up to some max
bound. This is especially important in the Quantum model as some states in a
superposition may require more loop iterations then others to evaluate.

So to implement a loop we want to repeatedly perform some computation until a
condition is met. We then want to stop performing that computation for the
remainder of the loop.

Swapping out of the loop using as done in conditional statements will not work
since some of the superposition states will reset the condition bit.

On solution currently implemented in the compiler is to only allow loops for
which the number of iterations is known at compile time. In that case it is
simple to implement since the circuit will just be repeated a known number of
times.

Alternatively if a conditional exit is required the counter described in
\cref{sec:findFirst} can incremented conditionally on the exit condition.
The body of the loop can then be controlled by checking against the expected
value of the counter for that iteration.
