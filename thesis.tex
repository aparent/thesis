\RequirePackage[l2tabu, orthodox]{nag}

\documentclass{report}

\usepackage{etex}
\usepackage{graphicx}
\usepackage{float}
\usepackage{subcaption}
\usepackage{circuitikz}

\usepackage{mathtools}

\usepackage{verbatim}

\usepackage{microtype}
\usepackage[colorlinks=false, pdfborder={0 0 0}]{hyperref}
\usepackage[all]{hypcap}
\usepackage{todonotes}

\usepackage{etex}
\reserveinserts{28}

\usepackage{graphicx}
\usepackage{float}
\usepackage{subcaption}
\usepackage{tikz,circuitikz,tikzscale}
\usepackage{pgfplots}
\usepackage{tikzscale}
\usepackage{rotating}

\usepackage{mathtools}
\usepackage{algorithm,algorithmicx,algpseudocode,amssymb,amsthm}
\usepackage{verbatim}
\usepackage{lmodern}

\usepackage{microtype}
\usepackage[colorlinks=false, pdfborder={0 0 0}]{hyperref}
\usepackage[all]{hypcap}
\usepackage{todonotes}
\usepackage{marginfix}

\usepackage{cleveref}

\usetikzlibrary{external}
\tikzexternalize[prefix=tikzPictures/]

% used with the assignemnt template
\let\underdot=\d % rename builtin command \d{} to \underdot{}
\renewcommand{\d}[2]{\frac{d #1}{d #2}} % for derivatives
\newcommand{\dd}[2]{\frac{d^2 #1}{d #2^2}} % for double derivatives
\newcommand{\pd}[2]{\frac{\partial #1}{\partial #2}}
\newcommand{\pdd}[2]{\frac{\partial^2 #1}{\partial #2^2}}
\newcommand{\gv}[1]{\ensuremath{\mbox{\boldmath$ #1 $}}}
\newcommand{\grad}[1]{\gv{\nabla} #1} % for gradient
\newcommand{\ket}[1]{{\left\vert{#1}\right\rangle}} % for Dirac bras
\newcommand{\bra}[1]{{\left\langle{#1}\right\vert}} % for Dirac kets
\newcommand{\braket}[2]{{\left\langle #1 \vphantom{#2} \right\vert\left.\!\! #2 \vphantom{#1} \right\rangle}} % for Dirac brackets

\let\underdot=\d % rename builtin command \d{} to \underdot{}
\renewcommand{\d}[2]{\frac{d #1}{d #2}} % for derivatives

\makeatletter
\renewcommand{\todo}[2][]{\tikzexternaldisable\@todo[#1]{#2}\tikzexternalenable}
\makeatother

\newtheorem{theorem}{Theorem}

%    Q-circuit version 2
%    Copyright (C) 2004  Steve Flammia & Bryan Eastin
%    Last modified on: 9/16/2011
%
%    This program is free software; you can redistribute it and/or modify
%    it under the terms of the GNU General Public License as published by
%    the Free Software Foundation; either version 2 of the License, or
%    (at your option) any later version.
%
%    This program is distributed in the hope that it will be useful,
%    but WITHOUT ANY WARRANTY; without even the implied warranty of
%    MERCHANTABILITY or FITNESS FOR A PARTICULAR PURPOSE.  See the
%    GNU General Public License for more details.
%
%    You should have received a copy of the GNU General Public License
%    along with this program; if not, write to the Free Software
%    Foundation, Inc., 59 Temple Place, Suite 330, Boston, MA  02111-1307  USA

% Thanks to the Xy-pic guys, Kristoffer H Rose, Ross Moore, and Daniel Müllner,
% for their help in making Qcircuit work with Xy-pic version 3.8.  
% Thanks also to Dave Clader, Andrew Childs, Rafael Possignolo, Tyson Williams,
% Sergio Boixo, Cris Moore, Jonas Anderson, and Stephan Mertens for helping us test 
% and/or develop the new version.

\usepackage{xy}
\xyoption{matrix}
\xyoption{frame}
\xyoption{arrow}
\xyoption{arc}

\usepackage{ifpdf}
\ifpdf
\else
\PackageWarningNoLine{Qcircuit}{Qcircuit is loading in Postscript mode.  The Xy-pic options ps and dvips will be loaded.  If you wish to use other Postscript drivers for Xy-pic, you must modify the code in Qcircuit.tex}
%    The following options load the drivers most commonly required to
%    get proper Postscript output from Xy-pic.  Should these fail to work,
%    try replacing the following two lines with some of the other options
%    given in the Xy-pic reference manual.
\xyoption{ps}
\xyoption{dvips}
\fi

% The following resets Xy-pic matrix alignment to the pre-3.8 default, as
% required by Qcircuit.
\entrymodifiers={!C\entrybox}

    % Defines Dirac notation. %7/5/07 added extra braces so that the commands will work in subscripts.
\newcommand{\qw}[1][-1]{\ar @{-} [0,#1]}
    % Defines a wire that connects horizontally.  By default it connects to the object on the left of the current object.
    % WARNING: Wire commands must appear after the gate in any given entry.
\newcommand{\qwx}[1][-1]{\ar @{-} [#1,0]}
    % Defines a wire that connects vertically.  By default it connects to the object above the current object.
    % WARNING: Wire commands must appear after the gate in any given entry.
\newcommand{\cw}[1][-1]{\ar @{=} [0,#1]}
    % Defines a classical wire that connects horizontally.  By default it connects to the object on the left of the current object.
    % WARNING: Wire commands must appear after the gate in any given entry.
\newcommand{\cwx}[1][-1]{\ar @{=} [#1,0]}
    % Defines a classical wire that connects vertically.  By default it connects to the object above the current object.
    % WARNING: Wire commands must appear after the gate in any given entry.
\newcommand{\gate}[1]{*+<.6em>{#1} \POS ="i","i"+UR;"i"+UL **\dir{-};"i"+DL **\dir{-};"i"+DR **\dir{-};"i"+UR **\dir{-},"i" \qw}
    % Boxes the argument, making a gate.
\newcommand{\meter}{*=<1.8em,1.4em>{\xy ="j","j"-<.778em,.322em>;{"j"+<.778em,-.322em> \ellipse ur,_{}},"j"-<0em,.4em>;p+<.5em,.9em> **\dir{-},"j"+<2.2em,2.2em>*{},"j"-<2.2em,2.2em>*{} \endxy} \POS ="i","i"+UR;"i"+UL **\dir{-};"i"+DL **\dir{-};"i"+DR **\dir{-};"i"+UR **\dir{-},"i" \qw}
    % Inserts a measurement meter.
    % In case you're wondering, the constants .778em and .322em specify
    % one quarter of a circle with radius 1.1em.
    % The points added at + and - <2.2em,2.2em> are there to strech the
    % canvas, ensuring that the size is unaffected by erratic spacing issues
    % with the arc.
\newcommand{\measure}[1]{*+[F-:<.9em>]{#1} \qw}
    % Inserts a measurement bubble with user defined text.
\newcommand{\measuretab}[1]{*{\xy*+<.6em>{#1}="e";"e"+UL;"e"+UR **\dir{-};"e"+DR **\dir{-};"e"+DL **\dir{-};"e"+LC-<.5em,0em> **\dir{-};"e"+UL **\dir{-} \endxy} \qw}
    % Inserts a measurement tab with user defined text.
\newcommand{\measureD}[1]{*{\xy*+=<0em,.1em>{#1}="e";"e"+UR+<0em,.25em>;"e"+UL+<-.5em,.25em> **\dir{-};"e"+DL+<-.5em,-.25em> **\dir{-};"e"+DR+<0em,-.25em> **\dir{-};{"e"+UR+<0em,.25em>\ellipse^{}};"e"+C:,+(0,1)*{} \endxy} \qw}
    % Inserts a D-shaped measurement gate with user defined text.
\newcommand{\multimeasure}[2]{*+<1em,.9em>{\hphantom{#2}} \qw \POS[0,0].[#1,0];p !C *{#2},p \drop\frm<.9em>{-}}
    % Draws a multiple qubit measurement bubble starting at the current position and spanning #1 additional gates below.
    % #2 gives the label for the gate.
    % You must use an argument of the same width as #2 in \ghost for the wires to connect properly on the lower lines.
\newcommand{\multimeasureD}[2]{*+<1em,.9em>{\hphantom{#2}} \POS [0,0]="i",[0,0].[#1,0]="e",!C *{#2},"e"+UR-<.8em,0em>;"e"+UL **\dir{-};"e"+DL **\dir{-};"e"+DR+<-.8em,0em> **\dir{-};{"e"+DR+<0em,.8em>\ellipse^{}};"e"+UR+<0em,-.8em> **\dir{-};{"e"+UR-<.8em,0em>\ellipse^{}},"i" \qw}
    % Draws a multiple qubit D-shaped measurement gate starting at the current position and spanning #1 additional gates below.
    % #2 gives the label for the gate.
    % You must use an argument of the same width as #2 in \ghost for the wires to connect properly on the lower lines.
\newcommand{\control}{*!<0em,.025em>-=-<.2em>{\bullet}}
    % Inserts an unconnected control.
\newcommand{\controlo}{*+<.01em>{\xy -<.095em>*\xycircle<.19em>{} \endxy}}
    % Inserts a unconnected control-on-0.
\newcommand{\ctrl}[1]{\control \qwx[#1] \qw}
    % Inserts a control and connects it to the object #1 wires below.
\newcommand{\ctrlo}[1]{\controlo \qwx[#1] \qw}
    % Inserts a control-on-0 and connects it to the object #1 wires below.
\newcommand{\targ}{*+<.02em,.02em>{\xy ="i","i"-<.39em,0em>;"i"+<.39em,0em> **\dir{-}, "i"-<0em,.39em>;"i"+<0em,.39em> **\dir{-},"i"*\xycircle<.4em>{} \endxy} \qw}
    % Inserts a CNOT target.
\newcommand{\qswap}{*=<0em>{\times} \qw}
    % Inserts half a swap gate.
    % Must be connected to the other swap with \qwx.
\newcommand{\multigate}[2]{*+<1em,.9em>{\hphantom{#2}} \POS [0,0]="i",[0,0].[#1,0]="e",!C *{#2},"e"+UR;"e"+UL **\dir{-};"e"+DL **\dir{-};"e"+DR **\dir{-};"e"+UR **\dir{-},"i" \qw}
    % Draws a multiple qubit gate starting at the current position and spanning #1 additional gates below.
    % #2 gives the label for the gate.
    % You must use an argument of the same width as #2 in \ghost for the wires to connect properly on the lower lines.
\newcommand{\ghost}[1]{*+<1em,.9em>{\hphantom{#1}} \qw}
    % Leaves space for \multigate on wires other than the one on which \multigate appears.  Without this command wires will cross your gate.
    % #1 should match the second argument in the corresponding \multigate.
\newcommand{\push}[1]{*{#1}}
    % Inserts #1, overriding the default that causes entries to have zero size.  This command takes the place of a gate.
    % Like a gate, it must precede any wire commands.
    % \push is useful for forcing columns apart.
    % NOTE: It might be useful to know that a gate is about 1.3 times the height of its contents.  I.e. \gate{M} is 1.3em tall.
    % WARNING: \push must appear before any wire commands and may not appear in an entry with a gate or label.
\newcommand{\gategroup}[6]{\POS"#1,#2"."#3,#2"."#1,#4"."#3,#4"!C*+<#5>\frm{#6}}
    % Constructs a box or bracket enclosing the square block spanning rows #1-#3 and columns=#2-#4.
    % The block is given a margin #5/2, so #5 should be a valid length.
    % #6 can take the following arguments -- or . or _\} or ^\} or \{ or \} or _) or ^) or ( or ) where the first two options yield dashed and
    % dotted boxes respectively, and the last eight options yield bottom, top, left, and right braces of the curly or normal variety.  See the Xy-pic reference manual for more options.
    % \gategroup can appear at the end of any gate entry, but it's good form to pick either the last entry or one of the corner gates.
    % BUG: \gategroup uses the four corner gates to determine the size of the bounding box.  Other gates may stick out of that box.  See \prop.

\newcommand{\rstick}[1]{*!L!<-.5em,0em>=<0em>{#1}}
    % Centers the left side of #1 in the cell.  Intended for lining up wire labels.  Note that non-gates have default size zero.
\newcommand{\lstick}[1]{*!R!<.5em,0em>=<0em>{#1}}
    % Centers the right side of #1 in the cell.  Intended for lining up wire labels.  Note that non-gates have default size zero.
\newcommand{\ustick}[1]{*!D!<0em,-.5em>=<0em>{#1}}
    % Centers the bottom of #1 in the cell.  Intended for lining up wire labels.  Note that non-gates have default size zero.
\newcommand{\dstick}[1]{*!U!<0em,.5em>=<0em>{#1}}
    % Centers the top of #1 in the cell.  Intended for lining up wire labels.  Note that non-gates have default size zero.
\newcommand{\Qcircuit}{\xymatrix @*=<0em>}
    % Defines \Qcircuit as an \xymatrix with entries of default size 0em.
\newcommand{\link}[2]{\ar @{-} [#1,#2]}
    % Draws a wire or connecting line to the element #1 rows down and #2 columns forward.
\newcommand{\pureghost}[1]{*+<1em,.9em>{\hphantom{#1}}}
    % Same as \ghost except it omits the wire leading to the left. 


\bibliographystyle{plain}
%----------------------------------------------------------------------------------------------------


\begin{document}
\listoftodos 

\title{Compiling Reversible Circuits}
\author{Alex Parent}
\date{\today}
\maketitle

\begin{abstract}
\end{abstract}
\tableofcontents

\chapter{Introduction}


%----------------------------------------------------------------------------------------------------
\chapter{Ancilla Management}

Any irreversible function can be made reversible using the ``Bennett Method''\cite{Bennett:73}.
Using this method the issue of ancilla management is avoided.

In order to generate 

\subsection{Janus}
\subsubsection{Modification Operators}
In the Janus programming language\cite{YG:2007} modification operators are available for operations that can be done in-place.
\subsubsection{If Conditions}

\subsection{Revs}\todo{A citation for this is needed.  Get this after arxiv paper is posted.}
The approach taken by Revs is complementary to to Janus.


%----------------------------------------------------------------------------------------------------

\chapter{Arithmetic}

A library of efficient commonly used arithmetic functions is needed for efficient circuit generation.
Further it is useful to have in-place functions whenever possible. 

\section{Addition}
  An addition circuit is a circuit which takes two $n$ bit integers $a = a_1a_2\dotsc a_n$ and $b = b_1b_2\dotsc b_n$ and returns the sum $a+b$. 
  \subsection{Classic Ripple} 
    The classic ripple addition method simply adds the bits in each column starting from the least significant bit.
    The output for that bit is the result modulo 2 and a carry is set to be added to the next column if the result is $>1$.

    To preform this algorithm we will need a circuit that takes three bits then calculates a carry and a sum.
    These can then be applied iteratively taking the two bits from the column of the numbers that are to be added as well as the previous carry.
    
    This can be done with a full adder circuit.
    A full adder is a circuit which takes input bits $a$, $b$, and $c$ then returns the sum ($a \oplus b \oplus c$) as well as the carry ($ab\oplus ac \oplus bc$).
    In figure~\ref{fig:classicalFA} an irreversible implementation can be seen.
    \begin{figure}[ht]
        \capstart
        \centering 
        \begin{circuitikz} \draw
            (3,0) node[xor port] (abxor) {}
            (6,-0.5) node[xor port] (abcxor) {}
            (6,-2) node[and port] (caband) {}
            (6,-3.5) node[and port] (aband) {}
            (8,-2.75) node[or port] (carryor) {}
            (0,0) node[anchor=east] (a) {$a$}
                to[short, o-*] (0.5,0)   
                -| (abxor.in 1)
                (0.5,0)  |- (aband.in 1)
            (0,-0.5) node[anchor=east] (b) {$b$}
                to[short, o-*] (1,-0.5)    
                (1,-0.5) -| (abxor.in 2)
                (1,-0.5) |- (aband.in 2)
            (0,-1) node[anchor=east] (c) {$c$}
                to[short, o-*] (1.5,-1)   
                (1.5,-1) |- (abcxor.in 2)
                (1.5,-1) |- (caband.in 2)
            (abxor.out) to[short,-*] (3.5,0)
                (3.5,0) -| (abcxor.in 1)
                (3.5,0) |- (caband.in 1)
            (aband) -| (carryor.in 2)
            (caband) -| (carryor.in 1)
            (abcxor.out) node[anchor=west] {Sum}
            (carryor.out) node[anchor=west] {Carry}
            ;
        \end{circuitikz}
        \caption{Irreversible full adder}
        \label{fig:classicalFA}
    \end{figure}
    In the reversible case a circuit (see figure~\ref{fig:reversibleFA}) can be constructed that preforms the mapping: 
    \[
        (a,b,c,0) \mapsto (a,b,a\oplus b\oplus c,ab\oplus ac \oplus bc)
    \]  
    \begin{figure}[ht]
        \capstart
        \centering 
        \[
          \Qcircuit @C=1em @R=.7em {
              \lstick{a} & \ctrl{3} & \ctrl{2} & \qw      & \qw      & \rstick{a}\qw\\
              \lstick{b} & \qw      & \qw      & \ctrl{2} & \ctrl{1} & \rstick{b}\qw\\
              \lstick{c} & \ctrl{1} & \targ    & \ctrl{1} & \targ    & \rstick{a  \oplus b  \oplus c}\qw\\
              \lstick{0} & \targ    & \qw      & \targ    & \qw      & \rstick{ab \oplus ac \oplus bc}\qw
          }
        \]
        \caption{Reversible full adder}
        \label{fig:reversibleFA}
    \end{figure}
    These can then be chained together as they are in the irreversible case.  
    The result will be a circuit which takes inputs $(a,b)$ and produces outputs $(a,b,a+b)$.

\subsection{In-Place Ripple}
    It is sometimes the case that one of the input values is unneeded after the operation.
    In that case it would save space to overwrite the value.
    To do this we could try to construct adder which maps inputs $(a,b)\mapsto(a,a+b)$.

    Such an adder is described in Cuccaro et al.\cite{CDKM:2004} (Note that not all the optimizations described in the original paper are desirable with our gate set as we wish to maximize shared controls). 
    It is very similar to the classic ripple.
    The main improvement is the realization that information about the carry could be stored in one of the input bits of each column.
    This allows it to overwrite one of it's inputs with the resulting sum.
    The circuit uses two main operations: MAJ, and UMA (See figure~\ref{fig:majuma}).
    The MAJ operation takes an input carry $c$ and two input bits $a$ and $b$.
    It computes the carry onto $a$ and partially computes the sum on $b$, it leaves $c$ in an unclean state to be fixed later. 
    This circuit can be repeated and rippled through all of the columns.
    After that is done the UMA operation can be preformed.
    This operation cleans up $a$ and $c$ then finishes computing the sum on $b$. 

    \begin{figure}[ht]
        \capstart
        \centering 
        \begin{subfigure}{.45\textwidth}
            \centering 
            \[
              \Qcircuit @C=1em @R=.7em {
                 \lstick{c} & \qw          & \targ      & \ctrl{2} & \rstick{c \oplus a}\qw\\
                 \lstick{b} & \targ        & \qw        & \ctrl{1} & \rstick{b \oplus a}\qw\\
                 \lstick{a} & \ctrl{-1}    & \ctrl{-2}  & \targ    & \rstick{ab \oplus ac \oplus bc}\qw 
              }
            \]
            \caption{MAJ}
        \end{subfigure}
        \begin{subfigure}{.45\textwidth}
            \centering 
            \[
              \Qcircuit @C=1em @R=.7em {
                  & \ctrl{2} & \targ      & \ctrl{1} & \qw\\
                  & \ctrl{1} & \qw        & \targ    & \qw\\
                  & \targ    & \ctrl{-2}  & \qw      & \qw 
              }
            \]
            \caption{UMA}
        \end{subfigure}
        \caption{}
        \label{fig:majuma}
    \end{figure}

\section{Converting OR to XOR}\todo{Reference revs paper}
In the case of mutually exclusive statements XOR is equivalent to OR.
That is to say $a \lor b = a \oplus b$ if $a = 1 \implies b = 0$ and $b = 1 \implies a =0$.  
For example $a\land b \lor \neg a \land c = a\land b \oplus \neg a \land c$. 

This is very useful as it allows us to avoid the use of Toffoli gates and use less ancilla.
For example if we wished to compute $a \lor b \lor c$ we might use the circuit:

  \[
    \Qcircuit @C=1em @R=.7em {
        \lstick{a} & \ctrlo{3} & \qw      & \ctrlo{3} & \rstick{a}\qw\\
        \lstick{b} & \ctrlo{2} & \qw      & \ctrlo{2} & \rstick{b}\qw\\
        \lstick{c} & \qw       & \ctrlo{2}& \qw       & \rstick{c}\qw\\
        \lstick{0} & \targ     & \ctrl{1} & \targ     & \rstick{0}\qw \\
        \lstick{0} & \qw       & \targ    & \targ      & \rstick{a \lor b \lor c}\qw
    }
  \]

Where as $a \oplus b \oplus c$ can be computed as:

\[
    \Qcircuit @C=1em @R=.7em {
        \lstick{a} & \ctrl{3}  & \qw      & \qw      & \rstick{a}\qw\\
        \lstick{b} & \qw       & \ctrl{2} & \qw      & \rstick{b}\qw\\
        \lstick{c} & \qw       & \qw      & \ctrl{1} & \rstick{c}\qw\\
        \lstick{0} & \targ     & \targ    & \targ    & \rstick{a \oplus b \oplus c}\qw \\
    }
\]

Or if one of $a$, $b$, or $c$. Are not needed in the later computation it can be done in-place as:

 \[
    \Qcircuit @C=1em @R=.7em {
        \lstick{a} & \ctrl{2}  & \qw      & \rstick{a}\qw\\
        \lstick{b} & \qw       & \ctrl{1} & \rstick{b}\qw\\
        \lstick{c} & \targ     & \targ    & \rstick{a \oplus b \oplus c}\qw \\
    }
\]

Given a set of AND expressions that are combined using OR we want to find sets of mutually exclusive statements that minimize the use of AND.
We consider each AND expression to be a vertex on a graph and add edges between vertices that are mutually exclusive.
Now we cover this graph using a minimum number of cliques.

After finding these cliques each set of mutually exclusive statements can be implemented by evaluating the AND statements and combining all of the values on a single ancilla using XOR for each clique.
These results can then be combined using OR statements (since OR is associative it does not matter how we group the statements). 

This algorithm has been implemented and tested on the BLIF benchmarks. \todo{add reference}
The results are shown below:
\todo{chart}


%----------------------------------------------------------------------------------------------------

\chapter{jcc}
\section{Implementation}
\subsection{Gate Set}

The gate set used by the compiler includes the NOT, CNOT, and Toffoli gates.

When choosing the implementation of various operations consideration is given to the expansion into the Clifford+T gate set. More specifically to the minimization of T-gates. For example shared controls on Toffoli gates are desired as they result in T cancellation.

\subsection{Addition}
Addition is done using the CDKM\cite{CDKM:2004} adder as shown below:

\verbatiminput{jcc-examples/add.j}
\includegraphics[width=\textwidth]{images/add.pdf}

The final carry bit is not computed in this adder.
The result is an adder which computes $a+b \mod 2^n$

\subsection{Subtraction}
Subtraction can be done simply by reversing the addition circuit.

\subsection{Multiplication}
Multiplication is done with a simple shift and add circuit.
An adder adding the variables $a$ and $b$ consists of a CDKM\cite{CDKM:2004} adder controlled on each bit of a.
Each adder adds $b$ to some ancilla shifted left by the position of the control in a.
Each adder is smaller then the last as the multiplication is preformed $\mod 2^n$.
Below is an example of the compiler output for a multiplication:

\verbatiminput{jcc-examples/mult.j}
\includegraphics[width=\textwidth]{images/mult.pdf}

\subsection{Conditional Statements}
Conditional if-else branches can be evaluated by swapping the bits to the correct circuit path controlled on the if conditional. 
The other circuit path is evaluated on a set of ancilla bits is initialized to $H^{\otimes n}\ket{0}$.
Since this is an eigenvector of every permutation matrix it will be unchanged and can be cleaned up by reversing the initialization.

\verbatiminput{jcc-examples/ifExample.j}
\includegraphics[width=\textwidth]{images/ifExample.pdf}

Note that the assertion is a statement about the state of the program which is true if and only if the ``if'' branch is taken rather then the ``else'' branch. 
This is useful as it describes a property of our data that can be used to clean up the bit used to store the result of the conditional. 
i.e. If we know that we took that if branch rather then the else branch we can XOR the conditional bit and be assured it is being reset to zero.

\bibliography{thesis}

\end{document}
