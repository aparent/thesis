\chapter{Introduction}

\section{Landauer's principle}

This principle states that any logical irreversible operation must be
accompanied by a corresponding entropy increase in non-information bearing
degrees of freedom\cite{bennett03}. Another way of stating this is that
an operation which does not preserve information must have an associated energy
cost.

The minimum bound on the amount of energy required to preform an irreversible
operation is called the Landauer limit ($kT\ln 2$).

Reversible computing avoids this minimum cost by ensuring that no information is
erased by the computation.

\section{Reversible Circuits}

Consider a classical computation where we have inputs $a$, and $b$
and wish to know the output $a\land b$. Further the values of $a$, and $b$ are
not needed later in the computation. In this case we can calculate $(a,b)
\mapsto a \land b$, take the output $a\land b$, and discard the values of $a$,
and $b$. Since this function is not injective (ex. $0\land 0 = 0$ and $1 \land 0
= 0$) it is not reversible.

One way to make it reversible is to use a Toffoli gate. The Toffoli gate
implements the injective function $(a,b,c) \mapsto (a,b,a\land b \oplus c)$. If
$c$ is initialized to zero we have $(a,b,0) \mapsto (a,b,a\land b)$. Note that
this means the values of $a$ and $b$ remain on the input bits after the
computation.

Any irreversible function can be made reversible using the ``Bennett
Method''\cite{Bennett:73}. In this method each irreversible gate is replaced
with a reversible version with additional space allocated as necessary. Then the
final result is copied out to newly allocated set of bits and the entire circuit
is repeated in reverse. So given an irreversible function $f(x)$ (where $f(x)$
is of size $n$) we can use the Bennett method to generate a reversible function
$(x,0^{n+m}) \mapsto (x,f(x),0^m)$.


\section{Quantum Circuits}

\todo{ Discuss choice of Clifford plus T gate set and motivate with discussion
of surface code architecture}

\todo{ Discuss high cost of T-gate and comparatively low cost of Clifford gates
and non-local interactions.  }

\todo {Show decomposition of Toffoli gate to Clifford+T and discuss what this
means for circuit optimization.}

