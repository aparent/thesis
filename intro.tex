\chapter{Introduction}

The goal of this thesis is to describe a framework for the efficient
compilation of high-level programs into low level circuits. Also discussed will
be the use of pebble games in the optimization of reversible circuits.

\section{Reversible Computation}

When preforming computations information is regularly discarded. Actions such
as overwriting data cause many states to be mapped to one so even if one had
knowledge of the actions involved in a computation it may not be possible to
recover the initial state. These types of computations are therefore called
irreversible.

A reversible computation is a computation which preforms a 1-to-1 mapping
between states. In \cite{Bennett:73} Bennett showed that in principle any
computation could be made reversible (with a potentially large space overhead).

\section{Landauer's principle}

This principle\cite{landauer61} states that any logical irreversible operation
must be accompanied by a corresponding entropy increase in non-information
bearing degrees of freedom. Another way of stating this is that an operation
which does not preserve information must have an associated energy cost. This
can be understood through the second law of thermodynamics which states that
the entropy of a closed system cannot decrease. That is to say that the number
of possible states in which a closed system might be cannot decrease.

The minimum bound on the amount of energy required to preform an irreversible
operation is called the Landauer limit ($kT\ln 2$).

Reversible computing shows that it is in principle possible to avoid this cost
by ensuring that no information is erased by the computation.

\section{Reversible Circuits}

Due to the focus on generation of circuits for use in quantum computation we
focus on the circuit model. The primary feature of the circuit model that makes
in useful in this setting is the fact that all operations are applied
unconditionally. This is important in the case of a quantum computer with
classical control since in order for  the classical part to make decisions about
which operations to apply which are dependent on data a measurement has to be made.

Consider a classical computation where we have inputs $a$, and $b$ and wish to
know the output $a\land b$.  The classical AND gate computes $(a,b)\mapsto
(a\land b)$, discarding the values of $a$ and $b$.  Since this function is not
injective (ex. $0\land 0 = 0$ and $1 \land 0 = 0$) it is not reversible.

One way to make it reversible is to instead use the injective function
$(a,b,c) \mapsto (a,b,a\land b \oplus c)$. This is called the ``Toffoli Gate''.
(If $c$ is initialized to zero we have $(a,b,0) \mapsto (a,b,a\land b)$).

Any irreversible function can be made reversible using the ``Bennett
Method''\cite{Bennett:73}. In this method each irreversible gate is replaced
with a reversible version with additional space allocated as necessary. Then the
final result is copied out to newly allocated set of bits and the entire circuit
is repeated in reverse. So given an irreversible function $f(x)$ (where $f(x)$
is of size $n$) we can use the Bennett method to generate a reversible function
$(x,0^{n+m}) \mapsto (x,f(x),0^m)$.


\section{Quantum Circuits}

Analysis will be done using the Clifford plus T gate set.  Note that this gate
set does not include the Toffoli gate so it will have to be synthesised from
more primitive gates in the set.  This choice is made due it's availability in
error correction schemes such as the surface code\cites{}.\todo{Expand and
mention the cost of state distillation as motivation}

Long range interactions are assumed to be inexpensive in this model compared to
the cost of T gates.

\todo {Show decomposition of Toffoli gate to Clifford+T and discuss what this
means for circuit optimization.}
