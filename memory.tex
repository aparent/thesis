\chapter{Memory Management}

In a reversible computation memory management is somewhat complicated by the reversibility constraint.

For example consider a classical computation where we have inputs $a$, and $b$ and wish to know the output $a\land b$.
Further the values of $a$, and $b$ are not needed later in the computation.
In this case we can calculate $(a,b) \mapsto a \land b$, take the output $a\land b$, and discard the values of $a$, and $b$.
Since this function is not injective (ex. $0\land 0 = 0$ and $1 \land 0 = 0$) it is not reversible.

One way to make it reversible is to use a Toffoli gate.
The Toffoli gate implements the injective function $(a,b,c) \mapsto (a,b,a\land b \oplus c)$.
If $c$ is initialized to zero we have $(a,b,0) \mapsto (a,b,a\land b)$.
This means that we cannot simply discard the values of $a$ and $b$.

\section{Bennett}
Any irreversible function can be made reversible using the ``Bennett Method''\cite{Bennett:73}.
This result was intended to show that in principle any irreversible computation can be made reversible.


\section{Janus}
\subsection{Modification Operators}
In the Janus programming language\cite{YG:2007,LD:1982} modification operators are available for operations that can be done in-place.
For example the addition modification operator, \verb|+=|.

Operations that cannot be done in place can only occur to the right of an in place operator.
These operations can then be implemented,
the result can be applied in place to the left hand side value,
and then cleaned up using Bennett if necessary.
This means that memory need only be allocated temporarily for each modification statement.

\subsection{If Conditions}
If conditions require a pre-condition (used to choose which branch to take),
and a post-condition (true if and only if the top branch is taken).

These are very useful in the context of memory management.
The pre-condition can be used to set a bit which controls which operations are preformed.
The post condition can be used to clear this bit as the pre-condition may no longer be true after the loop is preformed.

\section{Revs}\todo{A citation for this is needed.  Get this after arxiv paper is posted.}
The approach taken by Revs is complementary to to Janus.


