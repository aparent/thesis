\chapter{Discussion and Open Problems}

A new game for Heuristic methods for playing pebble games on an MDD were
presented and pebble game analysis was applied to some arithmetic circuits of
interest resulting in an asymptotic space reduction for a class of recursive
functions (see \ref{sec:recpeb}). It is likely that there exist efficiently
computable methods with better time and/or space performance. 

Additionally methods for the conversion of other reversible languages to an MDD
intermediate might give further insight into the types of useful heuristic
strategies that might be implemented. More work should be done to explore MDD
games on some special graphs which correspond to common computations.

It might also be interesting to create a sort of graphical reversible
programming language where the programmer directly interacts with and
constructs an MDD. Such a language might make it easier to visualize the
structure imposed by reversibility. 

The MDD structure may also be useful as a compiler intermediate for reversible
languages since it seems general enough to be language agnostic and many
optimizations and time-space trade-offs can be done using the information
provided. This would allow optimizations provided by a good
$\text{MDD}\mapsto\text{circuit}$ compiler to be shared between languages. 
